\documentclass[12pt]{article}
\usepackage{amsmath}
\usepackage{bm}
\usepackage{graphicx}
\usepackage{geometry}
\usepackage{indentfirst}
\usepackage{amsfonts}
\geometry{legalpaper, portrait, margin=0.5in}
\usepackage{color}   %May be necessary if you want to color links
\usepackage{hyperref}
\hypersetup{
    colorlinks=true, %set true if you want colored links
    linktoc=all,     %set to all if you want both sections and subsections linked
    linkcolor=black,  %choose some color if you want links to stand out
}
\usepackage[siunitx]{circuitikz}
\usetikzlibrary{patterns}
\usetikzlibrary{decorations.markings}
\graphicspath{ {./images/} }
\usepackage{enumitem}
\usepackage{outlines}
\begin{document}
\newcommand*\dif{\mathop{}\!\mathrm{d}}

\setlist[enumerate]{noitemsep, topsep=0pt, itemindent=\parindent}
\setlist[itemize]{noitemsep, topsep=0pt, itemindent=\parindent}

\newenvironment{subitemize}
{ \begin{itemize}[leftmargin=\parindent] }
{ \end{itemize} }

\newenvironment{subenumerate}
{ \begin{enumerate}[leftmargin=\parindent] }
{ \end{enumerate} }

\newenvironment{nopagebr}
  {\par\nobreak\vfil\penalty0\vfilneg
   \vtop\bgroup}
  {\par\xdef\tpd{\the\prevdepth}\egroup
   \prevdepth=\tpd}

\pagebreak
\begin{outline}

\section*{Q1}

\textbf{A.} $\frac 23$

\textbf{B.} $E_2=[0,\frac 19]\cup[\frac 29,\frac 13]\cup[\frac 23,\frac 79]\cup[\frac 89,1]$
with length $\frac 49$.

\textbf{C.} $E_3=[0,\frac 1{27}]\cup[\frac2{27},\frac19]\cup[\frac29,\frac7{27}]\cup[\frac8{27},\frac13]\cup
[\frac23,\frac{19}{27}]\cup[\frac{20}{27},\frac79]\cup[\frac89,\frac{25}{27}]\cup[\frac{26}{27},1]$ with length $\frac8{27}$.

\textbf{D.} $\left(\frac23\right)^n$. In one step, $\frac13$ of the total length is removed; therefore, the new
length on step $n$ is $\frac23$ of the length on step $n-1$.

\section*{Q2}

Let $|X|$ denote the sum of interval lengths in set $X$.

Note that $E_{n+1}\subset E_n$; therefore, by intersect, $E=E_{\infty}$.
By Q1D, we have $|E_{\infty}|=\left(\frac23\right)^{\infty}$.

Using the geometric series formula $\sum_{n=1}^{\infty} ar^{n-1}$ to represent sum of lengths removed over all
steps, we can compute $|E_{\infty}|$.
When constructing $E_n$ ($n\ge1$), we remove $\frac13$ of $|E_{n-1}|=\left(\frac23\right)^{n-1}$.
Our series is then $\sum_{n=1}^{\infty} \frac13 \left(\frac23\right)^{n-1}$.
Using the formula for a geometric series $\frac{a}{1-r}$,
our expression evaluates to $\frac{\frac13}{1-\frac23}=1$. The length of $[0,1]$ is 1, and with
a length of 1 removed, the final length, $|E_{\infty}|$, is zero.

We conclude that $|E|=0$, since $E=E_{\infty}$.

\section*{Q3}

$\frac1{27},\frac1{81},\frac7{27},0,1,\frac13,\frac23$ are in the Cantor set.

\section*{Q4}

\textbf{A.} 0,1

\textbf{B.} 1,2

\textbf{C.} 020202

\textbf{D.} 010101

\section*{Q5}

We guess that the ternary expansion for $\frac14$ is repeating 02.

Let a "partition" at position $x$ ($0<x<1$) be a split of an interval into before $x$ and after $x$.
$\sum_{k=1}^n \frac{x_k}{3^k}$ is the position of the nearest partition in $I$ left of $x$ after $n$
steps of partitioning each interval in $I$ into thirds. For each partitioning step, suppose
we discard the two intervals in $I$ which $x$ is not in. The geometric series $\sum_{n=0}^{\infty} \frac23\left(\frac13\right)^n$,
which represents length removed from $I$, evalutes to 1, which means the final length of $I$ is zero, and
since $I$ always contains $x$ according to our partitioning process, $I=[x,x]$. This means that if the infinite
sum evaluates to $x$, we have the correct ternary expansion.

Note that $\sum_{k=1}^\infty \frac{x_k}{3^k}$ is just the sum of $\frac2{3^k}$ for even $k$. Thus,
we can write the infinite sum as a geometric series $\sum_{k=0}^\infty \frac29\left(\frac19\right)^k$.
Using the formula for a geometric series, we have $\frac{\frac29}{1-\frac19}=\frac14$.
We can then conclude that the ternary expansion for $\frac14$ is repeating 02.

Since we have determined the ternary expansion for $\frac14$ doesn't contain 1, $\frac14$ is in
the Cantor set; only $x$ values which have 1 in their ternary expansion are excluded.

We determined in Q4D that $\frac18$ has 1 in its ternary expansion, therefore it is excluded from the Cantor
set.

\section*{Q6}

Define $n_k$ ($k\ge 0$) to be the number of intervals in $E_k$. We use $E_0=I$, meaning $n_0=1$.

When constructing $E_k$ ($k\ge 1$), we split each interval in $E_{k-1}$ into its first and last third,
creating two intervals in $E_k$ from each interval in $E_{k-1}$.
Thus, $n_k=2n_{k-1}$. Since $n_0=1$, we have $n_k=2^k$.

The Cantor set is equivalent to $E_{\infty}$. Although each interval in the Cantor set has a finite number
of elements due to zero length, the number of intervals in the Cantor set is $n_{\infty}=2^{\infty}$, which is
uncountable.

\end{outline}
\end{document}
